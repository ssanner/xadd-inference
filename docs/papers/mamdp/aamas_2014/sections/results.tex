\section{Empirical Results}
\label{sec:results}

We evaluated the performance of our novel solution technique on an 
American option quitting game. The domain and results are presented
below.

\subsubsection{American Option Quitting Game}

American options are financial instruments which allow an investor to
bet on the outcome of a yes/no proposition. The proposition typically
relates to whether the price of a particular asset that underlies the option
will rise above or fall below a specified amount, known as the strike
price. When the option reaches maturity the investor receives a fixed
pay-off if their bet was correct and nothing otherwise. An American 
option can be exercised by an investor at any time during its lifetime.

We analyse the valuation of an American option as an extensive form
zero-sum game between an investor and the issuer of the option.
The problem has two states, one continuous variable $v \in \mathbb{R}$ for the 
market value of the option and one discrete variable $i \in \mathbb{R}$ for the 
investor's inventory of options. 

At each time step the investor has three binary actions, 
$b \in \left\{true, false\right\}$ to buy an option from the issuer, 
$s \in \left\{true, false\right\}$ to sell an option, and 
$h \in \left\{true, false\right\}$ to hold the option. 

The issuer has two binary actions, 
 $n \in \left\{true, false\right\}$ to offer a narrow bid-ask spread for the
option, and $w \in \left\{true, false\right\}$ to offer a wide spread.

The joint actions of the investor and issuer, $a_{\text{inv}}$ and 
$a_{\text{iss}}$, are assumed to have an impact on the market value
of the option. For simplicity we assume that the value may increase or
decrease by fixed step sizes given by $u$ and $-u$, respectively.

\begin{equation*}
P(v' | v, i, a_{\text{inv}}, a_{\text{iss}}) = \delta \left[ v' - \begin{cases}
      (c)  : & v \\ 
      (e) \wedge (i > 0): & v + u \\ 
      (b) \wedge (s) : & v - u \\ 
    \end{cases} \right] \\    
\end{equation*}

The transition function for inventory $i$ is 
\begin{equation*}
P(i' | v, i, a_{\text{inv}}, a_{\text{iss}}) = \delta \left[ i' - \begin{cases}
      (e) : & 0 \\ 
      (b) \wedge (s) : & i + 1 \\ 
      otherwise: & i \\ 
    \end{cases} \right] \\    
\end{equation*}

The Dirac function $\delta[\cdot]$ensures that the transitions are valid conditional
probability functions that integrate to 1.

The reward obtained by the investor at each time step is given by,
\begin{equation}
  R = 
    \begin{cases}
      (e) \wedge (v > \text{strike-price}) \wedge (i > 0) : & 1 \\ 
      (e) \wedge (v < \text{strike-price}) \wedge (i > 0) : & -1 \\ 
      otherwise: & 0 \\
    \end{cases} \nonumber
\end{equation}

