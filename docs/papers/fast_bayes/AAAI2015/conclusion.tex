\section{Conclusion}
\label{sect:conclusion}

In this work, we showed how to transform piecewise likelihoods in
graphical models for Bayesian inference into equivalent mixture models
and then provide a blocked Gibbs sampling approach for this augmented
model that achieves an \emph{exponential-to-linear} reduction in space
and time compared to a conventional Gibbs sampler.  Unlike rejection
sampling and baseline Gibbs sampling, the time complexity of the
proposed Augmented Gibbs method does not grow exponentially with the
amount of observed data and yields faster mixing times in high
dimensions than Metropolis-Hastings.

Future extensions of this work can also examine application of this
work to non-Bayesian inference models (i.e., general piecewise
graphical models).  For example, some clustering models can be
formalized as piecewise models with latent cluster assignments for
each datum -- the method proposed here allows linear-time Gibbs
sampling in such models.  To this end, this work opens up a variety of
future possibilities for efficient asymptotically unbiased (Bayesian)
inference in expressive piecewise graphical models that to date have
proved intractable or inaccurate for existing (Markov Chain) Monte
Carlo inference approaches.

%Clearly,
%the proposed method has its own short comings.  Being a variation of
%Gibbs sampler it also suffers from some of the shortcomings that Gibbs
%does: Some islands of high-probability states may be unreachable by
%Gibbs if there is no path between them.  This problem can be more
%problematic in the augmented version since limiting the number of
%active regions can introduce more none-traversable gaps.  Hopefully,
%these problems are not so common and can be avoided by running more
%than one Markov chain.  On the other hand, the experimental efficiency
%of the algorithm has been rather surprising to us.  We hope we have
%provided a viable toolkit for asymptotically unbiased reasoning on
%piecewise models.
