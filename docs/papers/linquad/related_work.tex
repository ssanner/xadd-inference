The most relevant vein of Related work is that of~\cite{feng04}
and~\cite{li05} which can perform exact dynamic programming on
DC-MDPs with rectangular piecewise linear reward and transition functions
that are delta functions.  While SDP can solve these same problems,
it removes both the rectangularity and piecewise restrictions on the
reward and value functions, while
retaining exactness.  
Heuristic search approaches with formal guarantees 
like HAO*~\cite{hao09} are an attractive future extension of SDP;
in fact HAO* currently uses the method of~\cite{feng04}, which could
be directly replaced with SDP.  While~\cite{penberthy94} has considered
general piecewise functions with linear boundaries (and in fact,
we borrow our linear pruning approach from this paper), this work
only applied to fully deterministic settings, not DC-MDPs.

Other work has analyzed limited DC-MDPS having only one continuous
variable.  Clearly rectangular restrictions are meaningless with
only one continuous variable, so it is not surprising that more
progress has been made in this restricted setting.  One continuous
variable can be useful for optimal solutions to time-dependent MDPs 
(TMDPs)~\cite{boyan01}.  Or phase transitions can be used to 
arbitrarily approximate one-dimensional continuous distributions
leading to a bounded approximation approach for arbitrary single continuous
variable DC-MDPs~\cite{phase07}.  
While this work cannot handle arbitrary stochastic
noise in its continuous distribution, it does exactly solve DC-MDPs
with multiple continuous dimensions.

Finally, there are a number of general DC-MDP approximation
approaches that use approximate linear programming~\cite{kveton06}
or sampling in a reinforcement learning style approach~\cite{munos02}.
In general, while approximation methods are quite promising in
practice for DC-MDPS, the objective of this paper was to push
the boundaries of \emph{exact} solutions; however, in some sense, 
we believe that more expressive exact solutions may also inform
better approximations, e.g., by allowing the use of data structures
with non-rectangular piecewise partitions that allow higher fidelity
approximations.
