%%%%%%%%%%%%%%%%%%%%%%%%%%%%%%%%%%%%%%%%%%%%%%%%%%%%%%%%%%%%%%%%%%%%%%%%%%
%%
%% NOTE: Better to put the graphics above the math... visualize the function,
%%       see the XADD representatin... then examine the mathematical semantics.
%%
\begin{figure}[t!]
\vspace{-5mm}
\begin{minipage}{0.30\textwidth}
\center
\vspace{-5mm}
\includegraphics[trim = 0cm 0cm 0cm 0cm,width=\textwidth]{Figures/stepfun/PiecewiseLinearEx2.pdf} 
\end{minipage}\hspace{-5mm}
\begin{minipage}{0.18\textwidth}
\center
\vspace{-5mm}
\includegraphics[width=\textwidth]{Figures/xadds/dia2.pdf}
\end{minipage}\\
\vspace{-10mm}

\begin{minipage}{0.16\textwidth}
{\scriptsize
\begin{align*}
f(x,y)= 
\begin{cases}
  \phi_1:& \hspace{-2mm} f_1\\ 
  \phi_2: & \hspace{-2mm}f_2\\ 
\end{cases}
\end{align*} }
\end{minipage}
\begin{minipage}{0.16\textwidth}
{\scriptsize
\begin{align*}
 \phi_1 & = \theta_{11}\vee \theta_{12}\\
\theta_{11} & = x\!<\!0 \\
\theta_{12} & = x\!>\!0 \wedge x\!<\!-y \\
 f_1 & = \displaystyle \frac{x}{2} 
\end{align*} }
\end{minipage}
\begin{minipage}{0.15\textwidth}
{\scriptsize
\begin{align*}
\phi_2 & = \theta_{21}\\
\theta_{21} & = x\!>\!0 \wedge y\!>\!x\\
f_2 & = \displaystyle \frac{x}{5} - \frac{y}{2}
\end{align*} }
\end{minipage}
\vspace{-6mm}

\caption{\footnotesize Example of piecewise linear function in case and XADD form: \emph{(top left)} plot of function $f(x,y)$; \emph{(top right)} XADD representing $f(x,y)$; \emph{(bottom)} case semantics for $f(x,y)$ demonstrating notation used in this work.}
\label{fig:symbex}
\end{figure}
%%%%%%%%%%%%%%%%%%%%%%%%%%%%%%%%%%%%%%%%%%%%%%%%%%%%%%%%%%%%%%%%%%%%%%%%%%

\label{sec:xadd}

We begin with a brief introduction to the extended algebraic decision
diagram (XADD), then in Section~\ref{sec:approx} we contribute
approximation techniques for XADDs.  In Section~\ref{sec:basdp}, we
will show how this approximation can be used in a bounded
approximate symbolic dynamic programming algorithm for hybrid MDPs.

\subsection{Case Semantics of XADDs}

%% THIS PAPER DOES NOT ALLOW POLYNOMIALS, SHOULD NOT MENTION THEM!

%% Need to make strong immediate connections between cases and
%% XADDs since XADDs have already been introduced and discuss why
%% you're taking the time to discuss the case representation.

An XADD is a function represented by a directed acyclic graph having a
fixed ordering of decision tests from root to leaf.  For example,
Figure~\ref{fig:symbex}(top left) shows the plot of a piecewise function and
\ref{fig:symbex}(top right) its XADD representation.  Underlying this
XADD is a simple piecewise linear
function that we can represent semantically in case form.
Specifically, given a domain of boolean and continuous variables
$(\vec{b}^T,\vec{x}^T) = ( b_1,\ldots,b_n,x_{1},\ldots,x_m )$, where
$b_i \in \{ 0,1 \}$ ($1 \leq i \leq n$) and $x_j \in
[x_j^{\min},x_j^{\max}]$ ($1 \leq j \leq m$) for
$x_j^{\min},x_j^{\max} \in \mathbb{R}$ and $x_j^{\max} > x_j^{\min}$,
a case statement representing an XADD with linear decisions and leaf
expressions takes the following piecewise linear form {\footnotesize
\begin{align}
f(\vec{b},\vec{x}) = 
\begin{cases}
  \phi_1(\vec{b},\vec{x}): & f_1(\vec{x}) \\ 
 \vdots&\vdots\\ 
  \phi_k(\vec{b},\vec{x}): & f_k(\vec{x}) \\ 
\end{cases} \, . \label{eq:case}
\end{align}
} 
%% The XADD could work with a full set of inequalities, it's not
%% worth discussing complications of only $<,>$ at this point in
%% the paper.  You want to get to Section 3 -- the contribution --
%% as soon as possible.
%%
%% The partial function details of XADDs do not play any role in
%% this paper, they only occur in the SDP backup and were covered
%% in previous work -- its irrelevant here.
Here the $f_i$ are linear expressions over $\vec{x}$ and the $\phi_i$
are logical formulae defined over the domain $(\vec{b}^T,\vec{x}^T)$ that
can include arbitrary logical ($\land,\lor,\neg$) combinations of (i)
boolean variables and (ii) linear inequalities over $\vec{x}$.

In the XADD example of Figure~\ref{fig:symbex},  
\emph{every leaf} represents a case value $f_i$ and 
\emph{every path from root to leaf} represents a conjunction
of decision constraints.  The disjunction of all path constraints
leading to leaf $f_i$ corresponds to a case partition
$\l \phi_i(\vec{b},\vec{x}) : f_i(\vec{x}) \r$.  Clearly, all case
partitions derived from an XADD must be mutually disjoint and
exhaustive of the domain $(\vec{b}^T,\vec{x}^T)$, hence XADDs
represent well-defined functions.

%% This is not important for XADD approximation... it has been
%% covered in previous work for SDP that is cited.
%
%The $\phi_i$ are required to be mutually disjoint from each other;
%however they may not exhaustively cover the state space, in which case
%$f$ is a \emph{partial function}, undefined for some variable
%assignments. Since they represent parts of the piecewise function,
%pairs $(f_i,\phi_i)$ are referred to as cases or partitions of $f$,
%moreover, these are identified with a piecewise function of a single
%partition, which is called a case function.

%% Because we only work with linear XADDs, make this restriction up
%% front as I've done above... then this entire paragraph collapses
%% to a short discussion of the polytopes.
%%
%% Theta is a really bad choice for polytopes since it almost always
%% entails a parameter vector in the AI literature.
%
%We are especially interested in piecewise linear functions and case
%linear functions, in which case all $f_i$ and all expressions in the
%inequalities in $\phi_i$ are linear. 
Since $\phi_i$ can be written in disjunctive normal form (DNF), i.e.,
$\phi_i \equiv \bigvee_{j=0}^{n_i} \theta_{ij}$ where $\theta_{ij}$
represents a conjunction of linear constraints over $\vec{x}$ and a
(partial) truth assignment to $\vec{b}$ corresponding to the $j$th path
from the XADD root to leaf $f_i$, we observe that every
$\theta_{ij}$ contains a bounded convex linear polytope ($\vec{x}$ is
finitely bounded in all dimensions as initially defined).  
We formally define the set of all convex linear polytopes contained in 
$\phi_i$ as
%\begin{equation*}
$C(\phi_i) = \{ \PT(\theta_{ij}) \}_j$, 
%\end{equation*}
where $\PT$ extracts the subset of linear constraints from
$\theta_{ij}$. Figure~\ref{fig:symbex}(top left) illustrates the different
polytopes in the XADD of \ref{fig:symbex}(top right) with
corresponding case notation in \ref{fig:symbex}(bottom).

\subsection{XADD Operations} 

XADDs are important not only because they compactly represent
piecewise functions that arise in the forthcoming solution of hybrid
MDPs, but also because operations on XADDs can efficiently exploit
their structure.  XADDs extend algebraic decision diagrams
(ADDs)~\cite{bahar93add} and thus inherit most unary and binary ADD
operations such as addition $\oplus$ and multiplication $\otimes$.
While the addition of two linear piecewise functions represented by
XADDs remains linear, in general their product may not (i.e., the values
may be quadratic); however, for the purposes of symbolic dynamic
programming later, we remark that we only ever need to multiply
piecewise constant functions by piecewise linear functions
represented as XADDs, thus yielding a piecewise linear result.

Some XADD operations do require extensions over the ADD, e.g., the
binary $\max$ operation represented here in case form:

\vspace{-3mm}
{\footnotesize
\begin{center}
\begin{tabular}{r c c c l}
&
\hspace{-7mm} $\max \Bigg(
  \begin{cases}
    \phi_1: \hspace{-2mm} & \hspace{-1mm} f_1 \\ 
    \phi_2: \hspace{-2mm} & \hspace{-1mm} f_2 \\ 
  \end{cases}$
$,$
&
\hspace{-4mm}
  $\begin{cases}
    \psi_1: \hspace{-2mm} & \hspace{-1mm} g_1 \\ 
    \psi_2: \hspace{-2mm} & \hspace{-1mm} g_2 \\ 
  \end{cases} \Bigg)$
&
\hspace{-4mm} 
$ = $
&
\hspace{-4mm}
  $\begin{cases}
  \phi_1 \wedge \psi_1 \wedge f_1 > g_1    : & \hspace{-2mm} f_1 \\ 
  \phi_1 \wedge \psi_1 \wedge f_1 \leq g_1 : & \hspace{-2mm} g_1 \\ 
  \phi_1 \wedge \psi_2 \wedge f_1 > g_2    : & \hspace{-2mm}f_1 \\ 
  \phi_1 \wedge \psi_2 \wedge f_1 \leq g_2 : & \hspace{-2mm} g_2 \\ 
  \hspace{1cm}\vdots \hspace{8mm} & \hspace{-2mm} \vdots
  \end{cases}$
\end{tabular}
\end{center}
\vspace{-1mm}
} While the $\max$ of two linear piecewise functions represented as
XADDs remains in linear case form, we note that unlike ADDs which
prohibit continuous variables $\vec{x}$ and \emph{only} have a \emph{fixed} set of
boolean decision tests $\vec{b}$, an XADD may need to create
\emph{new decision tests} for the linear inequalities $\{ f_1 > g_1, f_1 >
g_2, f_2 > g_1, f_2 > g_2 \}$ over $\vec{x}$ as a result of operations
like $\max$.

Additional XADD operations such as symbolic substitution, continuous
(action) parameter maximization, and integration required for the
solution of hybrid MDPs have all been defined
previously~\cite{sanner_uai11,zamani12} and we refer the reader to
those works for details.

%EDIT - one more operation...!%
%%
%% NOTE: This is quite a long explanation for what in the end is a somewhat
%%       obvious operation and it is another technical distraction from
%%       the main contribution.  This explanation should be compressed to
%%       a single sentence and mentioned later when it is required --
%%       although if you do a relative approximation, you need not discuss
%%       this at all.
%%
%Another operation defined for linear XADDs is the unary
%maximization. Since every leaf is a linear function and every path
%defines a restricting set of inequalities, a linear program solver can
%be called for each path to find the maximal value in that region. The
%discrete maximum of all the paths is the maximal value that can be
%reached in the function represented in the XADD. This is used both for
%obtaining the allowed approximation error from a relative allowed
%error and calculating the actual error as the maximum of the
%difference between the original function and the approximation.
